%!TEX program = xelatex
\documentclass[UTF8]{ctexart}
\usepackage{graphicx}
\usepackage{listings}
\usepackage{color}
\definecolor{lightgray}{rgb}{.9,.9,.9}
\definecolor{darkgray}{rgb}{.4,.4,.4}
\definecolor{purple}{rgb}{0.65, 0.12, 0.82}

\lstdefinelanguage{JavaScript}{
  keywords={typeof, new, true, false, catch, function, return, null, catch, switch, var, if, in, while, do, else, case, break},
  keywordstyle=\color{blue}\bfseries,
  ndkeywords={class, export, boolean, throw, implements, import, this},
  ndkeywordstyle=\color{darkgray}\bfseries,
  identifierstyle=\color{black},
  sensitive=false,
  comment=[l]{//},
  morecomment=[s]{/*}{*/},
  commentstyle=\color{purple}\ttfamily,
  stringstyle=\color{red}\ttfamily,
  morestring=[b]',
  morestring=[b]"
}

\lstset{
   language=JavaScript,
   backgroundcolor=\color{lightgray},
   extendedchars=true,
   basicstyle=\footnotesize\ttfamily,
   showstringspaces=false,
   showspaces=false,
   numbers=left,
   numberstyle=\footnotesize,
   numbersep=9pt,
   tabsize=2,
   breaklines=true,
   showtabs=false,
   captionpos=b
}

\title{基础平台 技术构成与模块说明}
\author{黑志强}
\date{\today}

\begin{document}
 \maketitle
 \tableofcontents



\section{base platform 相关模块 }
\begin{figure}[ht!]
\center{\includegraphics[width=\linewidth]{1.png}}
\end{figure}

edu-cas-client

edu-cas-server

edu-platform-interface  (jar)

edu-data-service (war)

edu-knowledge-hierarchy-interface

edu-knowledge-hierarchy-services (war)

edu-platform-webapp20151204 (war)

edu-open-api (war,给互动教学提供接口)

edu-statistics (edu-statistics-interface edu-statistics-service )

edu-platform-data-sysnc (区校文件同步)


\subsection{edu-platform-interface,edu-data-service}

\subsubsection{业务划分:}
权限管理:应用管理,资源管理,角色管理,用户管理,用户与学段学科关联(rbac)

组织信息:学校,年级,班级,教学计划,认知层次,能力体系

人员信息:老师,家长,学生,普通用户, 生源

课程数据:学段,学科,教材版本,册,学期

还包括,登录,密码,

\subsubsection{技术构成:}
groovy,gorm,spring boot,spring context,dubbo

工具:Apache commons-lang3,commons-beanutils,poi



\subsection{edu-knowledge-hierarchy-interface,edu-knowledge-hierarchy-services}
\subsubsection{业务划分:}
知识体系:知识点,教材及章节

    资源:电子教材,课外读物,教学资源,教学资源推送,教学资源报错

题库管理:试题查询(按照知识点和教材)

诊断计划:诊断计划管理,诊断计划,诊断计划审核

试题组卷:手工组卷,自动组卷,诊断计划组卷

成套试卷:成套试卷,试卷与章节关联

试题推送:试题推送,推送结果

试题遴选:试题审核,改派,审核结果,改派结果

校级管理:教学巡视,统计(备课,课堂,作业,检测,课堂反馈,活动,学生梳理,校本资源)

区级管理:统计(备课,作业,检测,课堂反馈,活动,学生梳理,校本资源,课堂)

还包含以下:数据推送,学校类型,消息公告,用户验证


\subsubsection{技术构成:}
alibaba druid(数据库连接池),mybatis,spring boot,spring context,dubbo

工具:guava,Apache commons-lang3,commons-beanutils,poi


\subsection{edu-open-api}
\subsubsection{业务划分:}
为互动教学,自主学习提供接口;

依赖 edu-knowledge-hierarchy-interface,edu-platform-interface
\subsubsection{技术构成:}
jboss resteasy(RESTful Web Services ,用于规范基于HTTP的RESTful Web Services的API),
mybatis,spring boot,spring context,dubbo

工具:guava,commons-beanutils,poi


\subsection{edu-platform-webapp}

\subsubsection{技术构成:}
前端基于ace template (ace基于bootstrap,ace本身集成了大量的常用的js插件),还用到了插件有bootstrap-tokenfield(input 框中 关键字),ztree,bootstrap-treeview(知识点树,章节树),toastmessage

后端:spring mvc, spring context,dubbo

工具:guava,Apache commons-lang3,commons-beanutils,poi(模板下载,学校,教师,学生,教材 下载用到),dox4j(试卷下载用到,对富文本支持比较好)

\begin{thebibliography}{1}

\bibitem{}
\newblock {\em resteasy}
\newblock http://resteasy.jboss.org/

\bibitem{}
\newblock {\em spring framework}
\newblock http://spring.io/

\bibitem{}
\newblock {\em mybatis}
\newblock https://github.com/mybatis/mybatis-3

\bibitem{}
\newblock {\em Apache Software Fundation}
\newblock http://www.apache.org/

\end{thebibliography}

\end{document}